\documentclass[a4paper]{article}
\usepackage[T1]{fontenc}
\usepackage[utf8]{inputenc}
\usepackage[italian]{babel}
\usepackage{lmodern}
\usepackage{amsmath} 
\usepackage{graphicx}
\usepackage{bm}
\usepackage{pdfpages}
\usepackage{esint} 
\let\oldoiint\oiint\renewcommand{\oiint}{\oldoiint\limits}

\begin{document}
\title{Campi Elettromagnetici}
\author{Cristiano Cuffaro}
\date{}
\maketitle

\section{Definizioni e relazioni fondamentali\\}

Spostamento dielettrico $\rightarrow \textbf{D} = \epsilon\,\textbf{E} = \epsilon_0\,\epsilon_r\,\textbf{E}$\\\\Campo magnetico $\rightarrow \textbf{H} = \frac{\textbf{B}}{\mu} = \frac{\textbf{B}}{\mu_0\,\mu_r}$\\

\emph{Equazioni di Maxwell}:

\begin{itemize}
\item $\oint_{s} \textbf{E·ds} = -\frac{d}{dt} \iint_{S} \textbf{B·n}_0 \,dS \Rightarrow \nabla\times\textbf{E} = -\frac{\partial\textbf{B}}{\partial t}$ (vortici di \textbf{E})

\item $\oint_{s} \textbf{H·ds} = \frac{d}{dt} \iint_{S} \textbf{D·n}_0 \,dS + \iint_{S}\textbf{J·n}_0 \,dS \Rightarrow \nabla\times\textbf{H} = \frac{\partial\textbf{D}}{\partial t} + \textbf{J}$ (vortici di \textbf{H})

\item$\oiint_{S}\textbf{n}_0\textbf{·D}\,dS = q \Rightarrow \nabla\textbf{·D} = \rho$ (sorgenti di \textbf{D})

\item$\oiint_{S}\textbf{n}_0\textbf{·B}\,dS = 0 \Rightarrow \nabla\textbf{·B} = 0$ (sorgenti di \textbf{B} nulle)
\end{itemize}

\subsection*{Corrente di conduzione}
Densità di corrente $\textbf{J}$ e corrispondente quantità integrale $\textbf{I}_S$ associata a densità di carica $\rho$ in moto con velocità media $\textbf{u}$.\\

$\textbf{J} = \rho\,\textbf{u}$;\hspace{10mm}$I_S = \iint_{S}\textbf{J·}\textbf{n}_0\,dS$\\\\
\emph{Equazione di continuità}: $\oiint_{S}\textbf{J·}\textbf{n}_0\,dS = \iiint_{V}\nabla\cdot\textbf{J}\,dV = -\frac{\partial}{\partial t}\iiint_{V}\rho\,dV$\\

in forma differenziale $\rightarrow \nabla\cdot\textbf{J} = -\frac{\partial\rho}{\partial t}$\\\\
Cariche mosse dal campo elettrico, secondo la conducibilità $g\,\,(S\cdot m^{-1})$ del materiale:\\\\

$\textbf{J} = g\,\textbf{E}\hspace{10mm} $(per un conduttore ideale $g\rightarrow\infty$)\\\\
\subsection*{Parametri del mezzo}
Il mezzo è caratterizzato magneticamente da:
\begin{itemize}
\item[-]costante dielettrica $\epsilon$;
\item[-]permittività magnetica $\mu$;
\item[-]conducibilità elettrica $g$;
\end{itemize}
Il mezzo si dice:
\begin{itemize}
\item\textbf{omogeneo} se i parametri non variano al variare della posizione (NON omogeneo $\Rightarrow$ il contrario);
\item\textbf{lineare} se ciascun parametro è indipendente dall'intensità dei campi;
\item\textbf{isotropo} se si comporta allo stesso modo in tutte le direzioni;\\\\
\underline{\textbf{Oss:}} Un mezzo \textbf{anisotropo} è invece caratterizzato da un'espressione tensoriale del parametro. Ad esempio:\\\\
per il parametro $\epsilon \rightarrow$\hspace{4mm}[\textbf{$\epsilon$}] $ = \begin{bmatrix}
\epsilon_{11} & \epsilon_{12} & \epsilon_{13}\\
\epsilon_{21} & \epsilon_{22} & \epsilon_{23}\\
\epsilon_{31} & \epsilon_{32} & \epsilon_{33}\\
\end{bmatrix}$\\
In un mezzo \emph{anisotropo} i vettori della coppia corrispondente (\textbf{E}, \textbf{D}; \textbf{H}, \textbf{B}; \textbf{E}, \textbf{J}) possono non essere paralleli tra loro. Trasformazioni lineari:\\
\begin{center}
\textbf{D} = \textbf{[$\epsilon$]$\,$E} ;\hspace{5mm}\textbf{B} = \textbf{[$\mu$]$\,$H} ;\hspace{5mm}\textbf{J} = \textbf{[$g$]$\,$E} ;
\end{center}
\item\textbf{chirale} quando i vettori elettrici e magnetici dipendono dai corrispondenti vettori di entrambi i tipi.
\end{itemize}
\subsection*{Grandezze impresse}
Se $\textbf{J} = g\,\textbf{E}$ :\\
\hspace*{30mm}$\nabla\times\textbf{E} = -\frac{\partial\textbf{B}}{\partial t}$
\begin{flushright}
sono equazioni omogenee\\
\end{flushright}
\hspace*{30mm}$\nabla\times\textbf{H} = -\frac{\partial\textbf{D}}{\partial t} + g\textbf{E}$\\\\
\emph{Chi ha generato il campo magnetico?}\\
$\Rightarrow$ \'E generato dai processi che trasformano energia di "altro tipo" in energia elettromagnetica.\\
Si parla quindi di \emph{corrente impressa}:\hspace{5mm}$\textbf{J}_i \neq g\textbf{E}$\\
le sorgenti impresse non derivano dalla presenza dei campi MA li generano.\\
\begin{center}
$\nabla\times\textbf{H} = \frac{\partial\textbf{D}}{\partial t} + g\textbf{E} + \textbf{J}_i$\\
\end{center}
Nella pratica la corrente impressa non descrive l'effettiva sorgente del campo ma una sorgente \emph{equivalente} fissata a priori per poter determinare il campo.\\
Invece, la \emph{corrente magnetica impressa}:\hspace{5mm}$\textbf{J}_{im}$ non è "fisica" ma "matematica".\\\\
Con le correnti di sorgente\\\\
\hspace*{30mm}$\nabla\times\textbf{E} = -\frac{\partial\textbf{B}}{\partial t} - \textbf{J}_m - \textbf{J}_{im}$\\\\
\hspace*{30mm}$\nabla\times\textbf{H} = \frac{\partial\textbf{H}}{\partial t} + \textbf{J} + \textbf{J}_i$\\\\
\underline{\textbf{N.B.}} $\textbf{J}_m$ è nulla e si scrive solo per motivi di simmetria.\\\\
Possiamo osservare che esistono queste \emph{dualità}:
\begin{center}
$\textbf{E}\rightarrow\textbf{H}\hspace{15mm}\textbf{H}\rightarrow -\textbf{E}$\\
$\textbf{J}\rightarrow\textbf{J}_m\hspace{15mm}\textbf{J}_m\rightarrow -\textbf{J}$\\
$\epsilon\longleftrightarrow\mu$\\
\end{center}
(Le trasformazioni NON alterano le soluzioni!!)\\
\subsection*{Condizioni al contorno}
Relazioni differenziali che costituiscono un vincolo lasco per le soluzioni, che sono classi di funizoni. La soluzione si trova imponendo le condizioni al contorno. Vincoli compatibili con le proprietà fisiche e vincoli in corrispondenza di superfici di separazione tra mezzi differenti.\\\\
%\underline{Esempio cilindro con asse normale alle superfici}\\\\
%\begin{figure}[ht] 
%\centering
%\includegraphics[width=0.7\linewidth]{im1}
%\end{figure}\\
%Per Gauss:\\
%\hspace*{15mm}$\oiint_{S}\textbf{D·}\textbf{n}_0\,dS = \iiint_{V}\rho\,dV = \iint_{S1}\textbf{D}_1\textbf{·n}_1\,dS + \dots + \iint_{S3}\textbf{D}_3\textbf{·n}_3\,dS$\\\\
%quando $\Delta h\rightarrow 0,\,\,S_3\rightarrow 0,\,\,S_1\rightarrow S_2\rightarrow S$ e $\textbf{n}_1\rightarrow -\textbf{n}_\rightarrow-\textbf{n}_0$\\
%\begin{equation*}
%\Rightarrow\,\,\iint_{S}(\textbf{D}_2-\textbf{D}_1)\textbf{·n}_0\,dS = \lim_{\Delta h\to 0}\iiint_{V}\rho\,dV\\
%\end{equation*}
%Due casi:
%\begin{enumerate}
%\item $\rho$ finita $\Rightarrow\iiint_{V}\rho\,dV$ svanisce
%\begin{center}$(\textbf{D}_2-\textbf{D}_1)\textbf{·n}_0 = 0$\end{center}
%\item $\rho = \sigma\delta(z-z_0)$ per la proprietà di campionamento della funzione impulsiva
%\begin{center}$(\textbf{D}_2-\textbf{D}_1)\textbf{·n}_0 = \sigma$\end{center}
%\end{enumerate}
%Dalle precedenti si ricava per dualità\begin{equation*}(\textbf{B}_2-\textbf{B}_1)\textbf{·n}_0 = \sigma_m = 0\end{equation*}
\begin{tabular}{||c||c||}
\hline
\hline
\textbf{Componenti normali}&\textbf{Componenti tangenziali}\\
\hline
\hline
$\textbf{n}_0\cdot(\textbf{E}_2-\frac{\epsilon_1}{\epsilon_2}\textbf{E}_1) = \frac{\sigma}{\epsilon_2}$&$\textbf{n}_0\times(\textbf{E}_2-\textbf{E}_1)=0$\\
\hline
$\textbf{n}_0\cdot(\textbf{D}_2-\textbf{D}_1)=\sigma$&$\textbf{n}_0\times(\textbf{D}_2-\frac{\epsilon_2}{\epsilon_1}\textbf{D}_1)=0$\\
\hline
$\textbf{n}_0\cdot(\textbf{H}_2-\frac{\mu_1}{\mu_2}\textbf{H}_1)=0$&$\textbf{n}_0\times(\textbf{H}_2-\textbf{H}_1)=\textbf{K}$\\
\hline
$\textbf{n}_0\cdot(\textbf{B}_2-\textbf{B}_1)=0$&$\textbf{n}_0\times(\textbf{B}_2-\frac{\mu_2}{\mu_1}\textbf{B}_1)=\mu_2\textbf{K}$\\
\hline\hline
\end{tabular}\\\\\\
dove \textbf{K} è la corrente superficiale di densità lineare ($Am^{-1}$) finita.
\section{Bilancio energetico e unicità}
\subsection*{Il teorema di Poynting}
\begin{equation*}
\oiint_{S}\textbf{E}\times\textbf{H}\cdot\textbf{n}_0\,dS+\iiint_{V}(\textbf{H}\cdot\frac{\partial\textbf{B}}{\partial t}+\textbf{E}\cdot\frac{\partial\textbf{D}}{\partial t})\,dV+\iiint_{V}g\textbf{E}\cdot\textbf{E}\,dV=\iiint_{V}(-\textbf{J}_i\cdot\textbf{E}-\textbf{J}_{im}\cdot\textbf{H})\,dV\\\\
\end{equation*}
Il termine $\iiint_{V}(-\textbf{J}_i\cdot\textbf{E}-\textbf{J}_{im}\cdot\textbf{H})\,dV$ rappresenta la potenza che le correnti impresse (\emph{sorgenti}) creano all'interno del volume $V$.\\\\
\underline{\textbf{Oss:}} l'integrando è $\neq0$ solo nei punti in cui $\textbf{J}_i\neq0$ e $\textbf{J}_{im}\neq0$; tali punti individuano il \emph{volume di sorgente}, che in generale non coincide con il volume, arbitrario, $V$.\\\\
Il teorema di Poynting mostra che la potenza creata dalle sorgenti si divide in tre parti, corrispondenti ai termini a primo membro:
\begin{enumerate}
\item $\iiint_{V}g\textbf{E}\cdot\textbf{E}\,dV=\iiint_{V}\textbf{J}\cdot\textbf{E}\,dV$\\
è la potenza che il campo cede alle correnti di conduzione e che viene trasformata in calore (dissipata) per effetto Joule;
\item $\iiint_{V}(\textbf{H}\cdot\frac{\partial\textbf{B}}{\partial t}+\textbf{E}\cdot\frac{\partial\textbf{D}}{\partial t})\,dV=\frac{\partial W_m}{\partial t}+\frac{\partial W_e}{\partial t}$\\
è la potenza che va a variare l'energia immagazzinata nel campo elettromagnetico;
\item $\oiint_{S}\textbf{E}\times\textbf{H}\cdot\textbf{n}_0\,dS$\\
è la potenza che fluisce attraverso la superficie $S$ che racchiude il volume V.
\end{enumerate}
$\textbf{E}\times\textbf{H}\rightarrow$ \emph{vettore di Poynting}, rappresenta la densità superficiale di potenza associata al campo elettromagnetico.\\\\
\subsection*{Applicazione a sorgenti armoniche}
Osserviamo il caso in cui le sorgenti, e di conseguenza i campi, variano sinusoidalmente nel tempo:\\\\
\hspace*{40mm}$\textbf{J}_i=J_i\sin(\omega t)\,\textbf{i}_0$\\
\hspace*{40mm}$\textbf{E}=E\sin(\omega t+\psi_e)\,\textbf{e}_0$\\
\hspace*{40mm}$\textbf{H}=H\sin(\omega t+\psi_h)\,\textbf{h}_0$\\\\
Consideriamo il teorema di Poynting e osserviamo che quando le grandezze hanno andamento periodico, più che i valori istantanei sono significativi i valori medi in un periodo T.
\subsubsection*{Mezzo non dissipativo}
\begin{equation*}
-\frac{1}{2}\iiint_{V}J_iE\,\textbf{i}_0\cdot\textbf{e}_0\,\cos\psi_e\,dV=\frac{1}{T}\int_0^T\oiint_{S}\textbf{E}\times\textbf{H}\cdot\textbf{n}_0\,dS\,dt\\
\end{equation*}
I termini corrispondenti a variazioni di energia immagazzinata sono a media nulla. Ipotizzando $\textbf{i}_0\cdot\textbf{e}_0>0$ si osserva che il segno del valore medio della potenza creata dipende dallo sfasamento $\psi_e$:
\begin{itemize}
\item se $\frac{\pi}{2}<\psi_e<\frac{3\pi}{2}$, la potenza creata è positiva (fuoriesce dalla superficie);
\item se $\psi_e=\frac{\pi}{2}$ o $\psi_e=\frac{3\pi}{2}$ ("quadratura"), la sorgente non eroga potenza, ma crea potenza periodica a media nulla;
\item se $0<\psi_e<\frac{\pi}{2}$ o $\frac{3\pi}{2}<\psi_e<2\pi$, la potenza creata risulta negativa $\Rightarrow$ non c'è una sorgente MA un elemento dissipativo.
\end{itemize}
\subsubsection*{Involucro metallico}
Consideriamo ora la sorgente racchiusa in un involucro conduttore ideale, all'interno del quale il mezzo è, in generale, dissipativo ($g\neq0$).\\Il campo elettrico su $S$ è normale alla superficie $\Rightarrow$ il flusso del vettore di Poynting attraverso $S$ è nullo.\\\\
Considerando le quantità medie su un periodo e assumendo $\textbf{i}_0\cdot\textbf{e}_0\neq0$ si ha:
\begin{equation*}
-\frac{1}{2}\iiint_{V}J_iE\,\textbf{i}_0\cdot\textbf{e}_0\,\cos\psi_e\,dV=\iiint_{V}g\frac{E^2}{2}\,dV\\
\end{equation*}
tutta la potenza erogata dalla sorgente si dissipa nel materiale.\\\\
\underline{\textbf{N.B.}} se il materiale fosse privo di dissipazioni ($g=0$) la fase verrebbe modificata in modo tale da avere sfasamento $\psi_e = \frac{\pi}{2}$ ("quadratura") e quindi potenza media nulla.
\section{Campi nel dominio della frequenza}
\subsection*{Notazioni complesse}
Campo elettromagnetico sinusoidale con pulsazione $\omega$:\\\\
$\textbf{E}(t)=E_x(t)\,\textbf{x}_0+E_y(t)\,\textbf{y}_0+E_z(t)\,\textbf{z}_0$\hspace{10mm}con\hspace{5mm}$E_x(t)=E_{0_x}\cos(\omega t+\phi_x)$\\\\
\underline{\textbf{def}}\hspace{4mm}Vettore campo complesso:\\\\
\hspace*{10mm}$\hat{\textbf{E}}=E_{0_x}\,e^{j\phi_x}\,\textbf{x}_0+E_{0_y}\,e^{j\phi_y}\,\textbf{y}_0+E_{0_z}\,e^{j\phi_z}\,\textbf{z}_0$\\\\
\hspace*{10mm}$=E_{0_x}\,(\cos\phi_x+j\sin\phi_x)\,\textbf{x}_0+\cdots+E_{0_z}\,(\cos\phi_z+j\sin\phi_z)\,\textbf{z}_0$\\\\
\hspace*{10mm}$=(E_{x_r}+jE_{x_j})\,\textbf{x}_0+(E_{y_r}+jE_{y_j})\,\textbf{y}_0+(E_{z_r}+jE_{z_j})\,\textbf{z}_0$\\\\
\hspace*{10mm}$=\textbf{E}_r+j\,\textbf{E}_j$\\\\
Di conseguenza
\begin{equation*}
\textbf{E}(t)=Re[\,\hat{\textbf{E}}\,e^{j\omega t}\,]=\textbf{E}_r\cos\omega t-\,\textbf{E}_j\sin\omega t
\end{equation*}
\subsection*{Polarizzazione}
L'estremo libero di $\textbf{E}(t)$ descrive in generale un'ellisse nel piano individuato da $\textbf{E}_r$ e $\textbf{E}_j$: il vettore $\textbf{E}(t)$ è polarizzato ellitticamente. In casi particolari l'ellisse degenera in
\begin{itemize}
\item circonferenza (quando $\textbf{E}_r\cdot\textbf{E}_j=0$ e $|\textbf{E}_r|=|\textbf{E}_j|$): il vettore è polarizzato circolarmente;
\item segmento di retta (quando $\textbf{E}_r\times\textbf{E}_j=0$): il vettore ha polarizzazione lineare (o rettilinea);
\end{itemize}
\emph{Parametri di polarizzazione}:\\
$\rightarrow\,\,$ angolo di inclinazione $\psi$: angolo tra l'asse maggiore dell'ellisse di polarizzazione e una direzione (in genere orizzontale) nel piano dell'ellisse.\\
$\rightarrow\,\,$ angolo di ellitticità $\chi=\pm\arctan\frac{E_{min}}{E_{max}}$ ("+" se è polarizzazione sinistra, "-" se è destra).\\\\
\underline{\textbf{N.B.}} posso sempre esprimere qualunque polarizzazione come somma di polarizzazioni lineari (non $\parallel$).
\subsection*{Costante dielettrica nel dominio della frequenza}
In riferimento a mezzi rarefatti non polari (gas), studiamo $\epsilon$ in funzione del momento di dipolo elettrico $P$ indotto per unità di volume:
\begin{equation*}
\epsilon=\epsilon_0\,(1+\chi)=\epsilon_0\,\left[1+\frac{P}{\epsilon_0E}\right]
\end{equation*}
Allora poiché $\textbf{D}(t)=\epsilon_0\textbf{E}(t)+\textbf{P}(t)$, nel dominio della frequenza:
\begin{equation*}
\textbf{D}(\omega)=\epsilon_0\textbf{E}(\omega)+\textbf{P}(\omega)=\epsilon\textbf{E}(\omega)
\end{equation*}
\begin{equation*}
\Rightarrow\,\,\epsilon(\omega)=\epsilon_0\left[1+\frac{P(\omega)}{\epsilon_0E(\omega)}\right]
\end{equation*}
\subsection*{Mezzi non polari con cariche vincolate}
\begin{equation*}
\textbf{P}=P\,\textbf{p}_0=q\ell\,\textbf{p}_0
\end{equation*}
Il moto del sistema di cariche si ottiene dall'equilibrio delle forze, assunte parallele ad $\textbf{E}$:
\begin{equation*}
F_i+F_s+F_r=q\,E(t)
\end{equation*}
dove
\begin{itemize}
\item $F_i = m\frac{d^2\ell}{dt^2}$\hspace{29mm}è la forza di inerzia;
\item $F_s=s\frac{d\ell}{dt}$\hspace{31mm}è la forza di smorzamento;
\item $F_r=c\,\ell$\hspace{33mm}è la forza di richiamo;
\item $q\,E(t)=q\,E_0\cos\omega t$\hspace{15mm}è la forza esercitata dal campo elettrico.
\end{itemize}
Dal bilancio delle forze si ricava l'equazione differenziale
\begin{equation*}
m\frac{d^2\ell}{dt^2}+s\frac{d\ell}{dt}+c\,\ell=q\,E_0\cos\omega t
\end{equation*}
passando alla notazione complessa si ottiene l'equazione algebrica
\begin{equation*}
-\omega^2\hat{\ell}+j\omega\frac{s}{m}\hat{\ell}+\frac{c}{m}\hat{\ell}=\frac{q}{m}\hat{E}
\end{equation*}
definendo $\alpha=\frac{s}{2m}$ (coefficiente di smorzamento) e $\omega_0=\sqrt{\frac{c}{m}}$ (pulsazione di risonanza), l'equazione diventa
\begin{equation*}
(-\omega^2+2j\omega\alpha+\omega_0^2)\,q\hat{\ell}=\frac{q^2}{m}\hat{E}
\end{equation*}
e fornisce il fasore $\hat{P}=q\hat{ell}$ del momento di dipolo indotto
\begin{equation*}
\hat{P}=\frac{q^2}{m}\frac{\hat{E}}{(\omega_0^2-\omega^2)+2j\alpha\omega}
\end{equation*}
per cui
\begin{equation*}
\epsilon(\omega)=\epsilon_0(1+\frac{\hat{P}}{\epsilon_0\hat{E}})=\epsilon_0(\epsilon'+j\epsilon'')
\end{equation*}
La costante dielettrica relativa è
\begin{equation*}
\epsilon'+j\epsilon''=1+\frac{q^2}{\epsilon_0m}\frac{(\omega_0^2-\omega^2)-2j\alpha\omega}{(\omega_0^2-\omega^2)^2+4\alpha^2\omega^2}
\end{equation*}
Rispetto alla pulsazione di risonanza $\omega_0$ si individuano tre campi di frequenza caratteristici:
\begin{enumerate}
\item "basse" ($\omega\ll\omega_0$) frequenze
\begin{equation*}
\epsilon'\simeq1+\frac{q^2}{\epsilon_0m\omega_0^2};\hspace{15mm}-\epsilon''\simeq\frac{q^2}{\epsilon_0m}\frac{2\alpha\omega}{\omega_0^4}\ll\epsilon'
\end{equation*}
$\epsilon$ è circa reale e indipendente da $\omega$;
\item "alte" ($\omega\gg\omega_0$) frequenze
\begin{equation*}
\epsilon'\simeq1-\frac{q^2}{\epsilon_0m\omega^2};\hspace{15mm}-\epsilon''\simeq\frac{q^2}{\epsilon_0m}\frac{2\alpha}{\omega^3}\ll\epsilon'
\end{equation*}
$\epsilon$ è ancora prevalentemente reale e ha una debole dipendenza dalla frequenza;
\item frequenze nell'intorno della risonanza $\omega\simeq\omega_0$
\begin{equation*}
\epsilon'+j\epsilon''\simeq1+\frac{q^2}{2\epsilon_0m\omega_0}\left[\frac{·\omega}{(·\omega)^2+\alpha^2}-j\frac{\alpha}{(·\omega)^2+\alpha^2}\right]
\end{equation*}
con $·\omega=\omega_0-\omega$.\\
\end{enumerate}
\underline{\textbf{Oss:}} $\epsilon''<0$ e $\alpha>0$ sempre!
\subsection*{Mezzi compositi: atmosfera}
L'atmosfera è composta prevalentemente da azoto (poco polarizzabile) e da ossigeno e vapore acqueo.\\
$N_{O_2}$ singoli modi di polarizzazione:
\begin{equation*}
\epsilon'(\omega)=\sum_{i=1}^{N_{H_2O}}\left[S'F'(\omega)\right]_i+\sum_{i=1}^{N_{O_2}}\left[S'F'(\omega)\right]_i+\overline{\epsilon'}
\end{equation*}
\begin{equation*}
\hspace*{4mm}\epsilon''(\omega)=\sum_{i=1}^{N_{H_2O}}\left[S''F''(\omega)\right]_i+\sum_{i=1}^{N_{O_2}}\left[S''F''(\omega)\right]_i+\overline{\epsilon''}
\end{equation*}
Osserviamo che la parte immaginaria della costante dielettrica nasce quando siamo in presenza di dissipazioni, come in questo caso.
\begin{figure}[ht] 
\centering
\includegraphics[width=0.7\linewidth]{im8}
\caption{Parte reale e immaginaria della costante dielettrica dell'atmosfera in condizioni \emph{standard}. Si noti che $|\epsilon'-1|$ e $|\epsilon''|$ descrescono con la quota.}
\end{figure}
\subsection*{Mezzi conduttori}
Qui vi sono delle cariche libere di muoversi (corrente di conduzione) e c'è dissipazione.\\
La costante dielettrica relativa nel dominio della frequenza è data da
\begin{equation*}
\epsilon'+j\epsilon''=1-\frac{q^2}{\epsilon_0m}\frac{1}{\omega^2+4\alpha^2}-j\frac{q^2}{\epsilon_0m}\frac{2\alpha}{\omega(\omega^2+4\alpha^2)}
\end{equation*}
mentre invece la conducibilità complessa nel dominio della frequenza risulta
\begin{equation*}
g(\omega)=\frac{q^2}{m(2\alpha+j\omega)}=\frac{q^2}{m}\frac{2\alpha}{4\alpha^2+\omega^2}-j\frac{q^2}{m}\frac{\omega}{4\alpha^2+\omega^2}
\end{equation*}
a "bassa" ($\omega\ll\alpha$) frequenza
\begin{equation*}
g(\omega)\simeq\frac{q^2}{m}\frac{1}{2\alpha}-j\frac{q^2}{m}\frac{\omega}{4\alpha^2}
\end{equation*}
$|\mathrm{Im}[g]|\ll\mathrm{Re}[g]$.\\
Per un conduttore la parte immaginaria della costante dielettrica del mezzo dissipativo è pari a
\begin{equation*}
\epsilon''=-\frac{\mathrm{Re}[g]}{\omega\epsilon_0}
\end{equation*}
descrive lo stesso processo che descrive la parte reale di $g$, per cui
\begin{equation*}
\epsilon''\leftrightarrow\mathrm{Re}[g]
\end{equation*}
per mezzi di tipo diverso (non conduttori) come i mezzi condensati è comunque utile considerare questo tipo di relazioni:
\begin{equation*}
\epsilon''\simeq-\frac{g}{\omega\epsilon_0}\hspace{5mm}\Rightarrow\hspace{5mm}g_e=-\omega\epsilon_0\epsilon''
\end{equation*}
dove $g_e$ è la conducibilità equivalente del mezzo.
\section{Relazioni nel dominio della frequenza}
\subsection*{Teorema di Poynting}
\begin{equation*}
\iiint_V(-\frac{\textbf{J}_i^*\cdot\textbf{E}}{2}-\frac{\textbf{J}_{im}\cdot\textbf{H}^*}{2})\,dV
\end{equation*}
\begin{equation*}
=\iiint_Vg\frac{\textbf{E}\cdot\textbf{E}^*}{2}\,dV+j\omega\iiint_V(\mu\frac{\textbf{H}\cdot\textbf{H}^*}{2}-\epsilon^*\frac{\textbf{E}\cdot\textbf{E}^*}{2})\,dV+\frac{1}{2}\oiint_S(\textbf{E}\times\textbf{H}^*)\cdot\textbf{n}\,dS
\end{equation*}
Identifichiamo il significato dei vari termini:
\begin{itemize}
\item termine di sorgente
\begin{equation*}
\iiint_V(-\frac{\textbf{J}_i^*\cdot\textbf{E}}{2}-\frac{\textbf{J}_{im}\cdot\textbf{H}^*}{2})\,dV
\end{equation*}
assumendo polarizzazioni lineari nel dominio della frequenza e quindi, assunti $\textbf{J}_i$ ed $\textbf{E}$ concordi, $\textbf{i}_0\cdot\textbf{e}_0=1$
\begin{equation*}
\Rightarrow\hspace{5mm}-\frac{\textbf{J}_i^*\cdot\textbf{E}}{2}=-\frac{1}{2}J_0E_0\,\cos\psi_e-j\frac{1}{2}J_0E_0\,\sin\psi_e
\end{equation*}
\begin{itemize}
\item[-] parte reale: potenza media erogata nell'unità di volume dalle sorgenti elettriche;
\item[-] parte immaginaria: potenza reattiva che rappresenta la misura della potenza a media nulla, fornita e recuperata dalle sorgenti.
\end{itemize}
\item Il termine
\begin{equation*}
\iiint_Vg\frac{\textbf{E}\cdot\textbf{E}^*}{2}\,dV
\end{equation*}
è una quantità reale che coincide con la potenza media dissipata in un periodo per effetto della conducibilità.
\item L'integrando del termine
\begin{equation*}
j\omega\iiint_V(\mu\frac{\textbf{H}\cdot\textbf{H}^*}{2}-\epsilon^*\frac{\textbf{E}\cdot\textbf{E}^*}{2})\,dV
\end{equation*}
si può riscrivere individuando
\begin{itemize}
\item parte reale
\begin{equation*}
\frac{1}{2}\omega\iiint_V(\mu_0|\mu''|\textbf{H}\cdot\textbf{H}^*+\epsilon_0|\epsilon''|\textbf{E}\cdot\textbf{E}^*)\,dV
\end{equation*}
rappresenta la potenza media in un periodo dissipata per polarizzazione dielettrica e magnetica;
\item parte immaginaria
\begin{equation*}
\frac{1}{2}\omega\iiint_V(\mu_0\mu'\textbf{H}\cdot\textbf{H}^*+\epsilon_0\epsilon'\textbf{E}\cdot\textbf{E}^*)\,dV
\end{equation*}
misura l'ampiezza della variazione di energia immagazzinata nel campo elettrico e magnetico.
\end{itemize}
\item Il termine
\begin{equation*}
\frac{1}{2}\oiint_S(\textbf{E}\times\textbf{H}^*)\cdot\textbf{n}\,dS
\end{equation*}
è in generale complesso
\begin{itemize}
\item parte reale: potenza media su un periodo che fluisce attraverso la superficie $S$;
\item parte immaginaria: potenza reattiva sulla superficie $S$.\\\\
\underline{\textbf{Oss:}} $\bm{\mathcal{P}}=\frac{1}{2}\textbf{E}\times\textbf{H}^*$ è il vettore di Poynting complesso, densità superficiale di potenza ($Wm^{-2}$ o $VAm^{-2}$).
\end{itemize}
\end{itemize}
\subsection*{Bilancio energetico}
Parte della potenza erogata dal generatore si perde per dissipazione e in parte esce (fluisce fuori) $\Rightarrow$ il totale è uguale alla potenza erogata, detta \emph{potenza irradiata}.\\
Parte della potenza ceduta dalle correnti torna alle sorgenti bilanciando le variazioni periodiche di energia immagazzinata nei campi e l'eventuale rientro periodico di potenza attraverso $S$.\\\\
\underline{\textbf{Oss:}} se il mezzo è privo di dissipazioni ($g=0$ e $\epsilon$, $\mu$ reali) $\Rightarrow$ la potenza media erogata dalle sorgenti viene tutta irradiata attraverso $S$.
\section{Propagazione in mezzi non dissipativi}
Mezzo privo di dissipazioni, vale:
\begin{equation*}
\nabla^2\textbf{E}+k^2\textbf{E}+\nabla\left(\textbf{E}\cdot\frac{\nabla\epsilon}{\epsilon}\right)=0
\end{equation*}
\begin{flushright}
con $k^2=\omega^2\mu_0\epsilon$
\end{flushright}
Mezzo debolmente disomogeneo $|\nabla\epsilon|\rightarrow0$ si riduce ad una equazione delle onde omogenea a coefficiente non costante
\begin{equation*}
\nabla^2\textbf{E}+k^2(\textbf{r})\textbf{E}=0
\end{equation*}
L'equazione vale in modo approssimato per qualsiasi coppia $\nabla\epsilon$ e $\omega$ tali che
\begin{equation*}
\left|\frac{\nabla\epsilon}{\epsilon}\right|\ll k^2=\omega^2\mu_0\epsilon
\end{equation*}
\emph{Definizioni}
\begin{itemize}
\item $k_0=\omega\sqrt{\mu_0\epsilon_0}$ costante relativa al vuoto;
\item $n(\textbf{r})=\sqrt{\epsilon'(\textbf{r})}$ indice di rifrazione;
\item $k(\textbf{r})=n(\textbf{r})k_0$
\end{itemize}
Riscriviamo l'equazione delle onde in questa forma
\begin{equation*}
\nabla^2\textbf{E}+k_0^2n^2(\textbf{r})\textbf{E}=0
\end{equation*}
ipotizzando che abbia soluzione
\begin{equation*}
\textbf{E}(\textbf{r})=\textbf{E}_0e^{-jk_0\phi(\textbf{r})}
\end{equation*}
possiamo ricavare la \emph{equazione eiconale}
\begin{equation*}
n^2-|\nabla\phi|^2=0
\end{equation*}
\subsection*{L'onda elettromagnetica}
\begin{itemize}
\item[-] $\textbf{E}_0$ è il fattore determina ampiezza e polarizzazione;
\item[-] $e^{-jk_0\phi(\textbf{r})}$ è il fattore di fase. 
\end{itemize}
Campo
\begin{equation*}
\textbf{E}(\textbf{r},t)=Re\left[\textbf{E}(\textbf{r})e^{j\omega t}\right]=Re\left[\textbf{E}_0e^{-j[k_0\phi(\textbf{r})-\omega t]}\right]
\end{equation*}
\begin{equation*}
\Rightarrow E_{i}(r,t)=E_{0_i}\cos(k_0\phi(r)-\omega t)\hspace{15mm}i=x,y,z
\end{equation*}
\begin{figure}[ht] 
\centering
\includegraphics[width=0.7\linewidth]{im5}
\end{figure}
\\Se il tempo varia di $dt$, lo spostamento $dr$ lungo $\textbf{r}_0$ che annulla il differenziale è
\begin{equation*}
k_0(\nabla\phi)\cdot\textbf{r}_0dr-\omega dt=0
\end{equation*}
velocità di propagazione nella direzione $\textbf{r}_0$:
\begin{equation*}
\frac{dr}{dt}\biggl|_{\textbf{r}_0}=\frac{\omega}{k_0\nabla\phi\cdot\textbf{r}_0}=u\big|_{\textbf{r}_0}
\end{equation*}
La velocità $u$ dipede da $\textbf{r}_0$
\begin{itemize}
\item se $\textbf{r}_0\parallel\nabla\phi$
\begin{equation*}
u=\frac{\omega}{k_0|\nabla\phi|}=\frac{\omega}{k_0n}=\frac{c_0}{n}
\end{equation*}
con $c_0\simeq 3\cdot10^8\,\,m\cdot s^{-1}$\\
$u_{min}$ è la velocità di propagazione;
\item se $\textbf{r}_0\perp\nabla\phi$, $u\rightarrow\infty$ (non è velocità di trasporto).
\end{itemize}
\subsection*{Relazioni tra campi e direzione di propagazione}
Se $\textbf{E}=\textbf{E}_0e^{-jk_0\phi(\textbf{r})}$, anche $\textbf{H}=\textbf{H}_0e^{-jk_0\phi(\textbf{r})}$\\\\
Posto
\begin{itemize}
\item $\eta_0=\sqrt{\frac{\mu_0}{\epsilon_0}}$ \emph{impedenza intrinseca del vuoto};
\item $\eta=\sqrt{\frac{\mu}{\epsilon}}$ \emph{impedenza intrinseca del mezzo};
\end{itemize}
Dalle prime due equazioni di Maxwell
\begin{equation*}
\textbf{E}_0=-\eta\textbf{s}_0\times\textbf{H}_0
\end{equation*}
\begin{equation*}
\textbf{H}_0=\frac{1}{\eta}\textbf{s}_0\times\textbf{E}_0
\end{equation*}
componenti reali e immaginari di $\textbf{E}_0$ e $\textbf{H}_0$ sono ortogonali rispettivamente e alla direzione di propagazione $\textbf{s}_0$ e formano una terna trirettangola destra
\begin{equation*}
\textbf{h}_0=\textbf{s}_0\times\textbf{e}_0\,;\hspace{15mm}\textbf{e}_0=-\textbf{s}_0\times\textbf{h}_0
\end{equation*}
\begin{figure}[ht] 
\centering
\includegraphics[width=0.32\linewidth]{im2}
\end{figure}
\subsection*{Raggi elettromagnetici}
Vettore di Poynting
\begin{equation*}
\bm{\mathcal{P}}=\frac{1}{2}\textbf{E}\times\textbf{H}^*=\frac{1}{2}\textbf{E}_0\times\textbf{H}_0^*=\frac{1}{2}\textbf{E}_0\times\frac{\textbf{s}_0\times\textbf{E}_0^*}{\eta^*}=\frac{1}{2}\frac{\textbf{E}_0\cdot\textbf{E}_0^*}{\eta^*}\,\textbf{s}_0
\end{equation*}
ha direzione e verso di $\textbf{s}_0$, versore ortogonale alle superfici d'onda $\phi(\textbf{r})=cost$, indicando che il trasporto di potenza avviene ortogonalmente alle $\phi(\textbf{r})=cost$. Curve ortogonali in ogni punto alle superfici d'onda sono traiettorie dell'energia elettromagnetica, denominate \emph{raggi elettromagnetici}.\\\\
L’uso dei raggi elettromagnetici riduce il problema della propagazione da tridimen-
sionale a monodimensionale.\\
Va determinato il raggio (linea che congiunge la sorgente con il punto di ricezione)
e inoltre vanno determinate ampiezza, fase e polarizzazione del campo lungo il raggio.\\\\
\underline{\textbf{Oss:}} la traiettoria elettromagnetica dipende dalla distribuzione spaziale dell’indice di rifrazione e dalla direzione iniziale di propagazione.\\\\
Dall'equazione eiconale si ricava:
\begin{equation*}
\frac{d}{ds}(n\textbf{s}_0)=\textbf{s}_0\frac{dn}{ds}+n\frac{d\textbf{s}_0}{ds}=\nabla n
\end{equation*}
Per definizione, la curvatura di un raggio $\rho$ è data da $\frac{1}{\rho}$. Per quanto riguarda la curvatura del raggio elettromagnetico, è data da
\begin{equation*}
\frac{\textbf{n}_0}{\rho}=\frac{d\textbf{s}_0}{ds}
\end{equation*}
con $\textbf{n}_0$ normale principale della curva
\begin{equation*}
\frac{1}{\rho}=\textbf{n}_0\cdot\frac{\nabla n}{n}
\end{equation*}
la curvatura aumenta con $|\nabla n|$;\\
il raggio elettromagnetico rimane localmente rettilineo se $\textbf{s}_0\parallel\nabla n$.
\subsection*{Raggi in mezzi stratificati radialmente}
Consideriamo un mezzo il cui indice di rifrazione abbia simmetria sferica, ovvero vari solo con la coordinata radiale (è il caso della troposfera).
\begin{figure}[ht] 
\centering
\includegraphics[width=0.5\linewidth]{im3}
\end{figure}
\begin{equation*}
$$Dato che$$\nabla n=-|\nabla n|\textbf{R}_0
\end{equation*}
segue che
\begin{equation*}
\frac{1}{\rho}=\textbf{n}_0\cdot\frac{\nabla n}{n}=\frac{|\nabla n|}{n}\cos\psi=\frac{|\nabla n|}{n}\sin\theta
\end{equation*}
\subsection*{Principio di Fermat e lunghezza di percorso}
Un raggio elettromagnetico che passa per due punti $P_1$ e $P_2$ è tale che la lunghezza
$L$ del percorso elettromagnetico
\begin{equation*}
L=\int_{P_1}^{P_2}n(\textbf{r})ds
\end{equation*}
funzionale della traiettoria seguita tra $P_1$ e $P_2$, è stazionaria.\\
Considerata una qualunque curva che congiunge i due punti, quella (o quelle) che
rendono stazionario (generalmente minimo) il valore dell’integrale di linea dell’indice
di rifrazione è (sono) la/le traiettoria/e dell’energia elettromagnetica.
\paragraph*{Misura distanza}
\begin{equation*}
R=\frac{\tau\cdot c_0}{2}
\end{equation*}
$\tau$ tempo che l'energia elettromagnetica impiega tra 2 punti, la sua misura dà la distanza tra trasmettitore e ricevitore.
\section{Onde piane}
\subsection*{Onde piane in mezzi uniformi}
Nel caso di un mezzo uniforme e non dissipativo,
\begin{equation*}
\nabla\phi=n\textbf{s}_0=cost
\end{equation*}
e l'energia elettromagnetica si propaga lungo traiettorie rettilinee.\\
In un mezzo omogeneo la funzione eiconale vale
\begin{equation*}
\phi=n\int_{s_i}^sds=ns+cost
\end{equation*}
Per superfici d'onda $\phi=cost$ che siano dei piani (onde piane)
\begin{equation*}
\phi=\gamma_xx+\gamma_yy+\gamma_zz
\end{equation*}
L'onda piana è
\begin{equation*}
\textbf{E}(\textbf{r})=\textbf{E}_0e^{-j\textbf{k}\cdot\textbf{r}}
\end{equation*}
con $\textbf{k}=k_0\bm{\gamma}=k_0\nabla\phi$\\
i parametri del mezzo, la posizione delle sorgenti determinano posizione e verso.\\\\
Il \emph{vettore di propagazione} è in generale complesso
\begin{equation*}
\textbf{k}=\bm{\beta}-j\bm{\alpha}
\end{equation*}
e, più in generale, si distinguono due casi:
\begin{itemize}
\item mezzo senza dissipazioni,
\begin{equation*}
\textbf{k}\cdot\textbf{k}=\omega^2\mu\epsilon
\end{equation*}
e quindi necessariamente $-\alpha=0$ oppure $-\alpha\perp\beta$ e, inoltre, $\beta>\alpha$ ;
\item mezzo con dissipazioni (parametri complessi),\\\\
\begin{center}
$\alpha\neq0$ ;\hspace{15mm}$\alpha\not\perp\beta$
\end{center}
\end{itemize}
L'espressione generale di un'onda piana è dunque
\begin{equation*}
\textbf{E}(\textbf{r})=\textbf{E}_0e^{-j\textbf{k}\cdot\textbf{r}}=\textbf{E}_0e^{-j(\bm{\beta}-j\bm{\alpha})\cdot\textbf{r}}=\textbf{E}_0e^{-\bm{\alpha}\cdot\textbf{r}}\cdot e^{-j\bm{\beta}\cdot\textbf{r}}
\end{equation*}
in cui sono messi in evidenza
\begin{itemize}
\item \emph{fattore di ampiezza} $e^{-\bm{\alpha}\cdot\textbf{r}}$ ;
\item \emph{fattore di fase} $e^{-j\bm{\beta}\cdot\textbf{r}}$ ;
\end{itemize}
dove
\begin{itemize}
\item[-] $\bm{\alpha}$ individua i piani equiampiezza $\bm{\alpha}\cdot\bm{r}=cost$ ad esso ortogonali;
\item[-] $\bm{\beta}$ individua i piani equifase $\bm{\beta}\cdot\bm{r}=cost$ ad esso ortogonali.
\end{itemize}
Un'onda piana è \emph{uniforme} quando i piani equifase coincidono con quelli equiampiezza, cioè quando
\begin{equation*}
\bm{\alpha}=0\hspace{10mm}\text{oppure}\hspace{10mm}\bm{\alpha}\parallel\bm{\beta}
\end{equation*}
\begin{itemize}
\item \emph{velocità di fase} nella direzione $\textbf{r}_0$ ad angolo $\theta$ con $\bm{\beta}$
\begin{equation*}
u\big|_{\bm{r}_0}=\frac{\omega}{\bm{r}_0\cdot\bm{\beta}}=\frac{\omega}{\beta\cos\theta}
\end{equation*}
\item \emph{velocità di propagazione} nella direzione $\bm{\beta}$ (ha valore minimo), dipende dall'uniformità dell'onda:
\begin{itemize}
\item per un'onda uniforme
\begin{equation*}
u\big|_{\bm{\beta}_0}=\frac{\omega}{\omega\sqrt{\mu\epsilon}}=\frac{1}{\sqrt{\mu\epsilon}}=c
\end{equation*}
\item per un'onda non uniforme
\begin{equation*}
u\big|_{\bm{\beta}_0}=\frac{\omega}{\beta}=\frac{\omega}{\sqrt{\omega^2\mu\epsilon+\alpha^2}}<c
\end{equation*}
\end{itemize}
\end{itemize}
\subsection*{Relazioni tra campi e vettore di propagazione}
Sostituendo l'espressione dell'onda piana nelle prime due equazioni di Maxwell si ottiene
\begin{equation*}
\textbf{H}_0=\frac{\textbf{k}\times\textbf{E}_0}{\omega\mu}\,;\hspace{15mm}\textbf{E}_0=\frac{-\textbf{k}\times\textbf{H}_0}{\omega\epsilon}
\end{equation*}
con $\textbf{k}=\bm{\beta}-j\bm{\alpha}$.\\
Nel caso di onda piana uniforme $\textbf{k}=(\beta-j\alpha)\bm{\beta}_0=k\bm{\beta}_0$ e di conseguenza
\begin{equation*}
\textbf{H}_0=\frac{\bm{\beta}_0\times\textbf{E}_0}{\eta}\hspace{15mm}\textbf{E}_0=-\eta\bm{\beta}_0\times\textbf{H}_0
\end{equation*}
Osserviamo quindi che icomponenti di $\textbf{E}_0$ e $\textbf{H}_0$ di un'onda piana uniforme sono ortogonali
\begin{itemize}
\item[-] tra loro;
\item[-] al vettore di propagazione;
\item[-] al vettore di fase;
\end{itemize}
e, per un mezzo dissipativo, anche al vettore di attenuazione.
\subsection*{Costante di propagazione}
Per un'onda piana uniforme il parametro complesso $k=k_r-jk_j$ determina le caratteristiche di propagazione (progressione di fase) e di attenuazione.
\begin{itemize}
\item Mezzo dissipativo per conducibilità: $g\neq0$, $\epsilon$ e $\mu$ reali
\begin{equation*}
k=k_r-jk_j=\sqrt{-j\omega\mu(g+j\omega\epsilon)}
\end{equation*}
e le costanti di fase e attenuazione sono
\begin{equation*}
k_r=\omega\sqrt{\frac{\mu\epsilon}{2}\left[\sqrt{1+\left(\frac{g}{\omega\epsilon}\right)^2}+1\right]}
\end{equation*}
\begin{equation*}
k_j=\omega\sqrt{\frac{\mu\epsilon}{2}\left[\sqrt{1+\left(\frac{g}{\omega\epsilon}\right)^2}-1\right]}
\end{equation*}
\begin{itemize}
\item se $\frac{g}{\omega\epsilon}\ll 1$
\begin{equation*}
k_r\simeq\omega\sqrt{\mu\epsilon}\,;\hspace{15mm}k_j\simeq\frac{g}{2}\sqrt{\frac{\mu}{\epsilon}}
\end{equation*}
e il mezzo "si comporta da dielettrico";
\item se, invece, $\frac{g}{\omega\epsilon}\gg 1$
\begin{equation*}
k_r\simeq k_j\simeq\sqrt{\frac{\omega\mu\epsilon}{2}}
\end{equation*}
e il mezzo "si comporta da conduttore".
\end{itemize}
Per cui il comportamento del mezzo non dipende solo dal valore dei parametri $g$ ed $\epsilon$, ma anche dalla frequenza:
\begin{itemize}
\item[$\rightarrow$] "alta": il mezzo tende al dielettrico;
\item[$\rightarrow$] "bassa": il mezzo tende al conduttore.
\end{itemize}
\item Mezzo dielettrico dissipativo: $g=0$, $\epsilon$ complessa, $\mu$ reale
\begin{equation*}
k_r-jk_j=\omega\sqrt{\mu\epsilon_0(\epsilon'+j\epsilon'')}
\end{equation*}
se $|\epsilon''|\ll\epsilon'$
\begin{equation*}
k_r\simeq\omega\sqrt{\mu\epsilon_0\epsilon'}\,;\hspace{15mm}k_j\simeq\frac{\omega}{2}\sqrt{\mu\epsilon_0}\frac{|\epsilon''|}{\sqrt{\epsilon'}}
\end{equation*}
Dal punto di vista propagativo il mezzo può essere considerato privo di dissipazioni. Tuttavia, la presenza di $\omega$ a fattore nella costante di attenuazione può renderla molto elevata anche se $|\epsilon''|$ è bassa (è il caso della troposfera).
\end{itemize}
\begin{figure}[ht] 
\centering
\includegraphics[width=0.8\linewidth]{im13}
\caption{Attenuazione in decibel dovuta alla sola atmosfera su un percorso verticale tra un punto al livello del mare e un satellite allo zenit.}
\end{figure}
L'attenuazione (cui si riferisce il grafico) che subisce il campo nel percorso tra $s_i$ ed $s$, determinata dal rapporto tra i moduli dei campi o tra le densità di potenza, è comunemente espressa in dB:
\begin{equation*}
A(s,s_i)=-20\log_{10}\left|\frac{E(s)}{E(s_i)}\right|=-10\log_{10}\frac{\mathcal{P}(s)}{\mathcal{P}(s_i)}
\end{equation*}
\subsection*{Impedenza intrinseca}
Mezzo dissipativo per conducibilità: $g\neq 0$, $\epsilon$ e $\mu$ reali
\begin{equation*}
\eta=\eta_r+j\eta_j=\sqrt{\frac{\mu}{\epsilon}}\sqrt{\frac{1+j\frac{g}{\omega\epsilon}}{1+\left(\frac{g}{\omega\epsilon}\right)^2}}
\end{equation*}
\begin{itemize}
\item[-] se il mezzo "si comporta come dielettrico"
\begin{equation*}
\eta_r\simeq\sqrt{\frac{\mu}{\epsilon}}\,;\hspace{15mm}\eta_j\simeq\frac{g\sqrt{\mu}}{2\omega\epsilon^{\frac{3}{2}}}
\end{equation*}
\item[-] se il mezzo "si comporta come conduttore"
\begin{equation*}
\eta_r\simeq\eta_j\simeq\sqrt{\frac{\omega\mu}{2g}}
\end{equation*}
per cui se $g\rightarrow\infty\,$, $\eta\rightarrow0$.
\end{itemize}
\section{Riflessione e rifrazione delle onde piane}
\subsection*{Incidenza normale}
\subsubsection*{Materiale dielettrico}
\begin{figure}[ht] 
\centering
\includegraphics[width=0.4\linewidth]{im4}
\end{figure}
Due mezzi $M_1$ ed $M_2$ privi di dissipazioni con $\epsilon_1$ ed $\epsilon_2$ reali (di solito l'aria $\Rightarrow\epsilon_0,\,\mu_0$) ed $\epsilon_2,\,\mu_2$
\begin{equation*}
\textbf{E}(\textbf{r})=\textbf{E}_0e^{-j\beta z}\hspace{10mm}\textbf{E}'(\textbf{r})=\textbf{E}_0'e^{-j\beta'z}\hspace{10mm}\textbf{E}''(\textbf{r})=\textbf{E}_0''e^{-j\beta''z}
\end{equation*}
\\con
\begin{itemize}
\item $\bm{\beta}=\omega\sqrt{\mu\epsilon}\,\textbf{z}_0$\hspace{29mm}(incidente);
\item $\bm{\beta}'=\omega\sqrt{\mu_2\epsilon_2}\,\textbf{z}_0$\hspace{25mm}(rifratta);
\item $\bm{\beta}''=-\omega\sqrt{\mu\epsilon}\,\textbf{z}_0=-\bm{\beta}$\hspace{15mm}(riflessa);
\end{itemize}
da $M_1$ un'onda piana uniforme incide normalmente sul piano di separazione.\\\\
I campi sono determinati dalle ($z=0$)
\begin{equation*}
\textbf{E}_0+\textbf{E}_0''=\textbf{E}_0'\hspace{10mm}\textbf{H}_0+\textbf{H}_0''=\textbf{H}_0'
\end{equation*}
i campi rifratti e riflessi hanno gli stessi $\textbf{e}_0$ e $\textbf{h}_0$.\\\\
\begin{itemize}
\item \emph{Coefficiente di riflessione}:
\begin{equation*}
q_E=\frac{E_0''}{E_0}=\frac{\eta_2-\eta_1}{\eta_2+\eta_1}
\end{equation*}
\underline{\textbf{N.B.}}\hspace{5mm}$q_E=q_H$ (sono nulli quando $\eta_1=\eta_2$)
\item \emph{Coefficiente di trasmissione}:
\begin{equation*}
t_E=\frac{E_{0i}'}{E_{0i}}=1+\frac{\eta_2-\eta_1}{\eta_2+\eta_1}=\frac{2\eta_2}{\eta_2+\eta_1}
\end{equation*}
\begin{equation*}
t_H=\frac{H_{0i}'}{H_{0i}}=1-\frac{\eta_2-\eta_1}{\eta_2+\eta_1}=\frac{2\eta_1}{\eta_2+\eta_1}
\end{equation*}
\underline{\textbf{Oss:}}\hspace{5mm}$t_E=t_H=1$ se $\eta_1=\eta_2$
\end{itemize}
\includepdf[page={3}]{pdf}
\subsubsection*{Materiale conduttore}
\begin{equation*}
\eta_2=\sqrt{\frac{\omega\mu_2}{2g}}+j\sqrt{\frac{\omega\mu_2}{2g}}
\end{equation*}
e quindi
\begin{equation*}
\frac{\eta_2}{\eta_1}=\sqrt{\frac{\omega\mu_2}{2g}\cdot\frac{\epsilon_1}{\mu_1}}+j\sqrt{\frac{\omega\mu_2}{2g}\cdot\frac{\epsilon_1}{\mu_1}}
\end{equation*}
poiché in genere $\mu_2\simeq\mu_1,\,\,\epsilon_2$ è dello stesso ordine di grandezza di $\epsilon_1$
\begin{equation*}
\Rightarrow q_E\simeq-1,\hspace{5mm}q_H\simeq-1,\hspace{5mm}t_E\simeq 0
\end{equation*}
L'onda incidente viene riflessa quasi completamente.\\\\
Il campo elettrico in un materiale conduttore è praticamente nullo.\\
Il campo magnetico tangenziale sulla superficie del conduttore è il doppio di quello incidente ma decade esponenzialmente con la profondità.\\\\
Il campo elettromagnetico appena dentro il conduttore ideale è nullo.\\
All'esterno: sovrapposizione di onda incidente e riflessa
\begin{equation*}
\textbf{E}_{tot}=\textbf{E}_0e^{-j\beta z}+\textbf{E}_0''e^{j\beta z}=\textbf{E}_0(e^{-j\beta z}-e^{j\beta z})=-2j\textbf{E}_0\sin\beta z
\end{equation*}
\begin{equation*}
\textbf{H}_{tot}=\textbf{H}_0e^{-j\beta z}+\textbf{H}_0''e^{j\beta z}=\textbf{H}_0(e^{-j\beta z}+e^{j\beta z})=2\textbf{H}_0\cos\beta z
\end{equation*}
i campi non si propagano $\Rightarrow$ onde stazionarie.\\\\
Il vettore di Poynting del campo totale è
\begin{equation*}
\bm{\mathcal{P}}_{tot}=-j\frac{|E_0|^2}{\eta}\sin(2\beta z)\textbf{z}_0
\end{equation*}
è immaginario $\Rightarrow$ il campo totale non trasporta potenza.
\subsection*{Incidenza obliqua}
\subsubsection*{Materiale dielettrico}
Onda piana uniforme che incide con angolo $\theta$
\begin{figure}[ht] 
\centering
\includegraphics[width=0.4\linewidth]{im6}
\end{figure}
Campi
\begin{itemize}
\item incidenti
\begin{equation*}
\textbf{E}=\textbf{E}_0e^{-j(\beta_xx+\beta_zz)}
\end{equation*}
\begin{equation*}
\textbf{H}=\textbf{H}_0e^{-j(\beta_xx+\beta_zz)}
\end{equation*}
\item riflessi
\begin{equation*}
\textbf{E}''=\textbf{E}_0''e^{-j(\beta_x''x+\beta_z''z)}
\end{equation*}
\begin{equation*}
\textbf{H}''=\textbf{H}_0''e^{-j(\beta_x''x+\beta_z''z)}
\end{equation*}
\item rifratti
\begin{equation*}
\textbf{E}'=\textbf{E}_0'e^{-j(\beta_x'x+\beta_z'z)}
\end{equation*}
\begin{equation*}
\textbf{H}'=\textbf{H}_0'e^{-j(\beta_x'x+\beta_z'z)}
\end{equation*}
\end{itemize}
le condizioni di continuità dei componenti tangenziali $\textbf{E}_{0t}$ e $\textbf{H}_{0t}$ per $z=0$ richiedono che sia
\begin{equation*}
\textbf{E}_{0t}e^{-j\beta_xx}+\textbf{E}_{0t}''e^{\beta_x''x}=\textbf{E}_{0t}'e^{-j\beta_x'x}
\end{equation*}
\subparagraph*{Determinazione degli angoli di riflessione e rifrazione\\\\}
\'E necessario
\begin{equation*}
e^{-j\beta_xx}=e^{-j\beta_x''x}=e^{-j\beta_x'x}
\end{equation*}
\begin{equation*}
\Rightarrow\,\,\,\beta_x=\beta_x''=\beta_x'\,\,\,\Rightarrow\,\,\,
\begin{cases}
\text{\emph{legge di Erone}:}\hspace{4mm}\theta=\theta''\\
\text{\emph{legge di Snell}:}\hspace{5mm}\frac{\sin\theta}{\sin\theta'}=\frac{\beta'}{\beta}\\
\end{cases}
\end{equation*}
La legge di Snell impone l'uguaglianza delle velocità di fase lungo $x$ di onda incidente e rifratta
\begin{equation*}
\frac{\sin\theta}{\sin\theta'}=\frac{\beta'}{\beta}=\frac{\sqrt{\mu_2\epsilon_2}}{\sqrt{\mu_1\epsilon_1}}=\frac{u_1}{u_2}=\frac{n_2}{n_1}=n_{21}
\end{equation*}
\subparagraph*{Determinazione dei coefficienti di riflessione\\\\}
Sappiamo che il campo incidente può essere rappresentato da due vettori tra loro ortogonali e giacenti sul piano perpendicolare a $\bm{\beta}$
\begin{equation*}
\textbf{E}_0=\textbf{E}_{0h}+\textbf{E}_{0v}=E_{0y}\textbf{y}_0+E_{0v}\textbf{v}_0\hspace{10mm}\text{con}\hspace{5mm}\textbf{v}_0=\textbf{x}_0\cos\theta-\textbf{z}_0\sin\theta
\end{equation*}
Il campo magnetico corrispondente è
\begin{equation*}
\textbf{H}_0=\frac{\bm{\beta}_0\times\textbf{E}_0}{\eta_1}=-\frac{E_{0y}}{\eta_1}\textbf{v}_0+\frac{E_{0v}}{\eta_1}\textbf{y}_0
\end{equation*}
(espressioni analoghe si hanno per l'onda riflessa e per quella rifratta).\\\\
Pertanto, le condizioni al contorno proiettate sugli assi danno
\begin{equation*}
\begin{array}{lr}
E_{0y}+E_{0y}''=E_{0y}'\\
E_{0v}\cos\theta-E_{0v}''\cos\theta=E_{0v}'\cos\theta'\\
-\frac{E_{0y}}{\eta_1}\cos\theta+\frac{E_{0y}''}{\eta_1}\cos\theta=-\frac{E_{0y}'}{\eta_2}\cos\theta'\\
\frac{E_{0v}}{\eta_1}+\frac{E_{0v}''}{\eta_1}=\frac{E_{0v}'}{\eta_2}\\
\end{array}
\end{equation*}
\begin{itemize}
\item per la componente orizzontale del campo elettrico si ha
\begin{equation*}
\frac{\cos\theta}{\eta_1}(E_{0y}''-E_{0y})=-\frac{E_{0y}+E_{0y}''}{\eta_2}\cos\theta'
\end{equation*}
da cui
\begin{equation*}
\begin{array}{lr}
q_{Eh}=\frac{E_{0y}''}{E_{0y}}=\frac{\eta_2\cos\theta-\eta_1\cos\theta'}{\eta_2\cos\theta+\eta_1\cos\theta'}=\frac{\cos\theta-\frac{1}{n_{21}}\frac{\eta_1}{\eta_2}\sqrt{n_{21}^2-\sin^2\theta}}{\cos\theta+\frac{1}{n_{21}}\frac{\eta_1}{\eta_2}\sqrt{n_{21}^2-\sin^2\theta}}\,\,\,(\text{per la legge di Snell})\\\\
t_{Eh}=\frac{E_{0y}'}{E_{0y}}=\frac{2\eta_2\cos\theta}{\eta_2\cos\theta+\eta_1\cos\theta'}=\frac{2\cos\theta}{\cos\theta+\frac{1}{n_{21}}\frac{\eta_1}{\eta_2}\sqrt{n_{21}^2-\sin^2\theta}}\\
\end{array}
\end{equation*}
dove $n_{21}$ è l'indice di rifrazione relativo pari a $\frac{n_2}{n_1}$.\\\\
Per incidenza dall'aria su un dielettrico
\begin{equation*}
q_{Eh}\simeq\frac{\cos\theta-\sqrt{\epsilon_r-\sin^2\theta}}{\cos\theta+\sqrt{\epsilon_r-\sin^2\theta}}
\end{equation*}
per $\theta\rightarrow\frac{\pi}{2}\,\,\,\Rightarrow\,\,\,q_{Eh}\rightarrow0$;
\item per la componente verticale del campo elettrico si ha
\begin{equation*}
\eta_1(E_{0v}\cos\theta-E_{0v}''\cos\theta)=\eta_2(E_{0v}\cos\theta'+E_{0v}''\cos\theta')
\end{equation*}
da cui
\begin{equation*}
\begin{array}{lr}
q_{Ev}=\frac{E_{0v}''}{E_{0v}}=\frac{\eta_1\cos\theta-\eta_2\cos\theta'}{\eta_1\cos\theta+\eta_2\cos\theta'}=\frac{\cos\theta-\frac{1}{n_{21}}\frac{\eta_2}{\eta_1}\sqrt{n_{21}^2-\sin^2\theta}}{\cos\theta+\frac{1}{n_{21}}\frac{\eta_2}{\eta_1}\sqrt{n_{21}^2-\sin^2\theta}}\,\,\,(\text{per la legge di Snell})\\\\
t_{Ev}=\frac{E_{0v}'}{E_{0v}}=\frac{2\eta_2\cos\theta}{\eta_1\cos\theta+\eta_2\cos\theta'}=\frac{2\cos\theta}{\frac{\eta_1}{\eta_2}\cos\theta+\frac{1}{n_{21}}\sqrt{n_{21}^2-\sin^2\theta}}\\
\end{array}
\end{equation*}
Per incidenza dall'aria su un dielettrico
\begin{equation*}
q_{Ev}=\frac{\epsilon_r\cos\theta-\sqrt{\epsilon_r-\sin^2\theta}}{\epsilon_r\cos\theta+\sqrt{\epsilon_r-\sin^2\theta}}
\end{equation*}
$\exists\,\,\theta_B$ (angolo di Brewstel) tale per cui $q_{Ev}=0$ e
\begin{equation*}
n_{21}\eta_1\cos\theta_B=\eta_2\sqrt{n_{21}^2-\sin^2\theta_B}
\end{equation*}
Nell'ipotesi di mezzi dielettrici, $\mu_2\simeq\mu_1\simeq\mu_0$ e
\begin{equation*}
\cos\theta_B=\frac{\epsilon_1}{\epsilon_2}\sqrt{\frac{\epsilon_2}{\epsilon_1}-\sin^2\theta_B}\hspace*{5mm}\text{da cui}\hspace{5mm}\sin\theta_B=\sqrt{\frac{\epsilon_2}{\epsilon_1+\epsilon_2}}
\end{equation*}
\end{itemize}
\subparagraph*{Espressione delleonde riflessa e rifratta\\}
\begin{equation*}
\textbf{E}''=[q_{Eh}(\theta)\textbf{E}_{0h}+q_{Ev}(\theta)E_{0v}\textbf{v}_0'']e^{-j(\beta_xx-\beta_zz)}
\end{equation*}
\begin{equation*}
\textbf{E}'=[t_{Eh}(\theta)\textbf{E}_{0h}+t_{Ev}(\theta)E_{0v}\textbf{v}_0']e^{-j(\beta_x'x+\beta_z'z)}
\end{equation*}
con\hspace{5mm}$\textbf{v}_0''=-\textbf{x}_0\cos\theta-\textbf{z}_0\sin\theta$\hspace{5mm}e\hspace{5mm}$\textbf{v}_0'=\textbf{x}_0\cos\theta'-\textbf{z}_0\sin\theta'$.\\\\
Dato che $q_{Eh}\neq q_{Ev}$ tranne che per $\theta=0$ o $\theta\rightarrow\frac{\pi}{2}$, lo stato di polarizzazione dell'onda riflessa è in generale diverso da quello dell'onda incidente. In particolare, indipendentemente dalla polarizzazione dell'onda incidente, l'onda riflessa è polarizzata orizzontalmente quando $\theta=\theta_B$.
\includepdf[page={1}]{pdf}
\subsubsection*{Materiale dissipativo}
Mezzo dissipativo per effetto della conducibilità ($g\neq0$).\\
Campi incidente e riflesso
\begin{equation*}
\textbf{E}=\textbf{E}_0e^{-j(\beta_xx+\beta_zz)}
\end{equation*}
\begin{equation*}
\textbf{H}=\textbf{H}_0e^{-j(\beta_xx+\beta_zz)}
\end{equation*}
\begin{equation*}
\textbf{E}''=\textbf{E}_0''e^{-j(\beta_xx-\beta_zz)}
\end{equation*}
\begin{equation*}
\textbf{H}''=\textbf{H}_0''e^{-j(\beta_xx-\beta_zz)}
\end{equation*}
mentre il campo rifratto è caratterizzato dal vettore di propagazione complesso
\begin{equation*}
\textbf{E}'=\textbf{E}_0'e^{-\bm{\alpha}'\cdot\textbf{r}-j\bm{\beta}'\cdot\textbf{r}}
\end{equation*}
\begin{equation*}
\textbf{H}'=\textbf{H}_0'e^{-\bm{\alpha}'\cdot\textbf{r}-j\bm{\beta}'\cdot\textbf{r}}
\end{equation*}
Per la continuità, a $z=0$ si ha
\begin{equation*}
e^{-j\textbf{k}\cdot\textbf{r}}=e^{-j\textbf{k}'\cdot\textbf{r}}
\end{equation*}
ovvero
\begin{equation*}
e^{-j\beta_xx}=e^{-(\alpha_x'x+j\beta_x'x)}\hspace{10mm}\forall x
\end{equation*}
condizione sugli esponenti
\begin{equation*}
-j\beta_x=-\alpha_x'-j\beta_x'
\end{equation*}
\begin{equation*}
\alpha_x'=0\,;\hspace{15mm}\beta_x'=\beta_x
\end{equation*}
Di conseguenza
\begin{equation*}
\bm{\alpha}'=\alpha'\textbf{z}_0\,;\hspace{15mm}\beta_x'=\beta'\sin\theta'=\beta\sin\theta
\end{equation*}
Per cui
\begin{equation*}
\bm{\alpha}'\not\parallel\bm{\beta}'\hspace{5mm}\Rightarrow\hspace{5mm}\text{l'onda nel materiale conduttore \underline{non} è uniforme.}
\end{equation*}
\includepdf[page={2}]{pdf}
\subsection*{Riflessione totale}
Mezzi privi di dissipazioni.\\
Incidenza obliqua con la seguente ipotesi
\begin{equation*}
\sqrt{\mu_1\epsilon_1}>\sqrt{\mu_2\epsilon_2}
\end{equation*}
La continuità dei componenti tangenziali dei campi richiede che
\begin{equation*}
\sin\theta'=\frac{\sin\theta}{n_{21}}=\sqrt{\frac{\mu_1\epsilon_1}{\mu_2\epsilon_2}}\sin\theta
\end{equation*}
quindi per la nostra ipotesi notiamo che $\theta'>\theta$.
Ne consegue che all'aumentare dell'angolo di incidenza, la direzione di $\bm{\beta}'$ si allontana sempre di più dalla normale alla superficie di discontinuità, avvicinandosi ad $\textbf{x}_0$.\\\\
In particolare per $\theta=\theta_L$ (angolo limite)
\begin{equation*}
\sin\theta_L=\sqrt{\frac{\mu_2\epsilon_2}{\mu_1\epsilon_1}}
\end{equation*}
l'onda rifratta si propaga parallelamente al piano di discontinuità.\\
Oltrepassando $\theta_L$ la continuità dei componenti tangenziali dei campi non è più assicurata poiché la velocità di fase lungo $x$ dell'onda incidente è più bassa della minima velocità di fase che un'onda uniforme può avere nel secondo mezzo:
\begin{equation*}
\frac{\omega}{\beta\sin\theta}<\frac{\omega}{\beta'}\hspace{5mm}\text{per }\,\theta>\theta_L
\end{equation*}
Un'onda rifratta non uniforme ha velocità di fase
\begin{equation*}
u'=\frac{\omega}{\beta'}=\frac{\omega}{\sqrt{\alpha'^2+k'^2}}<\frac{1}{\sqrt{\mu_2\epsilon_2}}
\end{equation*}
Per il secondo mezzo privo di dissipazioni
\begin{equation*}
\textbf{E}'=\textbf{E}_0'e^{-\alpha'z-j\beta'x}
\end{equation*}
\begin{equation*}
\beta'=\sqrt{\alpha'^2+\omega^2\mu_2\epsilon_2}=\omega\sqrt{\mu_1\epsilon_1}\sin\theta
\end{equation*}
\begin{equation*}
\alpha'=\omega\sqrt{\mu_1\epsilon_1}\sqrt{\sin^2\theta-\frac{\mu_2\epsilon_2}{\mu_1\epsilon_1}}
\end{equation*}
Le espressioni dei campi nel secondo mezzo sono quindi
\begin{equation*}
\textbf{E}'=\textbf{E}_0'e^{-\omega\sqrt{\mu_1\epsilon_1}\left(\sqrt{\sin^2\theta-n_{21}^2}\,z+j\sin\theta\,x\right)}
\end{equation*}
\begin{equation*}
\textbf{H}_0'=\frac{(\beta'\textbf{x}_0-j\alpha'\textbf{z}_0)\times\textbf{E}_0'}{\omega\mu_2}
\end{equation*}
\begin{figure}[ht] 
\centering
\includegraphics[width=0.4\linewidth]{im7}
\end{figure}
\\l'onda del campo rifratto si propaga in direzione $\parallel$ al piano di discontinuità $\,\Rightarrow\,\,\theta'=\frac{\pi}{2}$
\subsubsection*{Trasporto di potenza}
Per comodità assumiamo l'onda incidente polarizzata orizzontalmente $\,\Rightarrow\,$ per continuità anche l'onda rifratta lo è
\begin{equation*}
\textbf{E}_0'=E_0'\textbf{y}_0
\end{equation*}
\begin{equation*}
\textbf{H}_0'=\frac{\beta'E_0'}{\omega\mu_2}\textbf{z}_0+j\frac{\alpha'E_0'}{\omega\mu_2}\textbf{x}_0
\end{equation*}
\begin{equation*}
\bm{\mathcal{P}'}=\frac{1}{2}j|E_0'|^2\frac{\alpha'}{\omega\mu_2}e^{-2\alpha'z}\textbf{z}_0+\frac{1}{2}|E_0'|^2\frac{\beta'}{\omega\mu_2}e^{-2\alpha'z}\textbf{x}_0
\end{equation*}
il termine diretto verso $\textbf{z}_0$ è immaginario mentre quello diretto verso $\textbf{x}_0$ è reale $\,\Rightarrow\,$ quando l'angolo di incidenza supera l'angolo limite non vi è flusso di potenza attraverso il piano di separazione ma la potenza fluisce parallelamente ad esso.
\section{Irradiazione}
Campo prodotto da una sorgente elettrica "puntiforme" in un mezzo omogeneo privo di dissipazioni
\begin{figure}[ht] 
\centering
\includegraphics[width=0.4\linewidth]{im10}
\end{figure}
\\i raggi elettromagnetici scaturiscono dal punto in cui è posta la sorgente.
\begin{equation*}
\textbf{E}(\textbf{r})=\textbf{E}_0e^{-jk_0\phi(r)}
\end{equation*}
con $\,\,\,\textbf{E}_0\perp\textbf{H}_0\perp\textbf{r}_0\parallel\bm{\mathcal{P}}$.\\
Nel mezzo omogeneo privo di dissipazioni $\phi(r)=nr\,\,\,\Rightarrow\,\,\,k_0\phi(r)=kr$ e considerata la generica sfera $S$ centrata sulla sorgente, per la conservazione dell'energia si ha
\begin{equation*}
Re\left[\oiint_S\frac{\textbf{E}\times\textbf{H}^*}{2}\cdot\textbf{n}\,dS\right]=Re\left[\oiint_S\frac{\textbf{E}_0\times\textbf{H}_0^*}{2}\cdot\textbf{n}\,dS\right]=cost\,\,\,\forall S
\end{equation*}
$\Rightarrow\,\,\,E_0$ e $H_0$ devono decrescere come $\frac{1}{r}$, inoltre i loro valori iniziali dipendono dalla corrente di sorgente.
\begin{figure}[ht] 
\centering
\includegraphics[width=0.4\linewidth]{im11}
\end{figure}
\\Assunta $\textbf{J}_i=J_i\textbf{z}_0$, se $J_i$ è indipendente da $\varphi$ (simmetria assiale), allora lo sono anche $\textbf{E}_0$ e $\textbf{H}_0$, ma dipendono in generale da $\theta$ oltre che da $r$.\\\\
Definito il \emph{momento della sorgente}:
\begin{equation*}
\emph{\textbf{M}}=\emph{M}\textbf{z}_0=\iiint_{V'}\textbf{J}_i(\textbf{r}')\,dV
\end{equation*}
il campo elettromagnetico irradiato è
\begin{equation*}
\textbf{E}=\frac{\eta\emph{M}}{2\pi r^2}\left(1+\frac{1}{jkr}\right)e^{-jkr}\cos\theta\,\textbf{r}_0
\end{equation*}
\begin{equation*}
\hspace*{17mm}+\frac{j\omega\mu\emph{M}}{4\pi r}\left(1+\frac{1}{jkr}-\frac{1}{k^2r^2}\right)e^{-jkr}\sin\theta\,\bm{\theta}_0
\end{equation*}
\begin{equation*}
\textbf{H}=\frac{jk\emph{M}}{4\pi r}\left(1+\frac{1}{jkr}\right)e^{-jkr}\sin\theta\,\bm{\phi}_0
\end{equation*}
proporzionale a \emph{M}.
\begin{itemize}
\item $\textbf{H}$ è ortogonale alla direzione della sorgente e alla direzione radiale;
\item $\textbf{E}$ giace in un piano meridiano, contenente la direzione della sorgente.
\end{itemize}
Le espressioni dei campi contengono somme di potenze di $\frac{1}{jkr}=\frac{1}{j\beta r}=\frac{1}{j2\pi}\frac{\lambda}{r}$ per cui, a seconda del rapporto $\frac{r}{\lambda}$
\begin{itemize}
\item[-] possono prevalere i termini di ordine massimo o minimo;
\item[-] può prevalere il componente meridiano di E rispetto a quello radiale.
\end{itemize}
\begin{itemize}
\item A distanza "piccola" (rispetto a $\lambda$) dalla sorgente
\begin{equation*}
E\propto\frac{1}{r^3}\hspace{15mm}H\propto\frac{1}{r^2}
\end{equation*}
come nel caso statico di dipolo elettrico o magnetico.
Il campo nelle “vicinanze” (rispetto a $\lambda$) della sorgente è detto campo di induzione.
Il campo di induzione può essere elevato anche se la potenza erogata dalla
sorgente è bassa (predomina la potenza reattiva).
\item A distanza "grande" (rispetto a $\lambda$) dalla sorgente
\begin{equation*}
\textbf{E}\simeq j\frac{\eta\emph{M}}{2\lambda r}e^{-j\beta r}\sin\theta\,\bm{\theta}_0\,;\hspace{5mm}\textbf{H}\simeq\frac{j\emph{M}}{2\lambda r}e^{-j\beta r}\sin\theta\,\bm{\phi}_0\,;\hspace{5mm}\textbf{H}=\frac{\textbf{r}_0\times\textbf{E}}{\eta}
\end{equation*}
il campo è detto \emph{campo di radiazione} e trasporta la potenza erogata (irradiata) dalla sorgente.\\\\
\emph{Campo di radiazione}
\begin{itemize}
\item è un'onda sferica la cui ampiezza decresce come $\frac{1}{r}$, come richiesto dalla conservazione dell'energia;
\item $\textbf{E}$ e $\textbf{H}$ sono trasversi tra loro e a $\textbf{r}_0$ (direzione del trasporto di energia e direzione locale di propagazione);
\item il rapporto tra $E$ e $H$ è l'impedenza intrinseca del mezzo.
\end{itemize}
\emph{Campo asintotico}
\begin{itemize}
\item ha le proporietà di un raggio elettromagnetico.
\end{itemize}
\end{itemize}
\subsection*{Irradiazione da sorgente corta filiforme}
Sorgente cilindrica corta e sottile di lunghezza $\ell$ in cui scorre corrente $I$, con $\textbf{J}_i$ indipendente dalle coordinate;\\
$\emph{M}=I\ell$;\\
il campo di radiazione è
\begin{equation*}
\textbf{E}_\infty(\textbf{r})=j\frac{\eta I}{2}\frac{\ell}{\lambda}\frac{e^{-j\beta r}}{r}sin\theta\,\bm{\theta}_0;\hspace{10mm}\textbf{H}_\infty=\frac{\textbf{r}_0\times\textbf{E}_\infty}{\eta}
\end{equation*}
la potenza irradiata è proporzionale a $\left(\frac{\ell}{\lambda}\right)^2$: se la sorgente è "corta" rispetto a $\lambda$ essa irradia poco.\\\\
Per il generatore, il filamento di corrente che irradia equivale a una resistenza $R_i$ (\emph{resistenza di radiazione}) che dissipa la potenza $W_i$ irradiata
\begin{equation*}
W_i=\frac{\pi}{3}\eta I^2\left(\frac{\ell}{\lambda}\right)^2
\end{equation*}
\begin{equation*}
R_i=\frac{2W_i}{I^2}=\frac{2}{3}\pi\eta\left(\frac{\ell}{\lambda}\right)^2
\end{equation*}
qualunque elemento di circuito percorso da corrente è una sorgente che irradia.\\\\
Per dimensioni dell’ordine del centimetro, almeno sino a frequenze dell’ordine delle centinaia di $MHz$ ($\lambda\simeq1\,m$), la radiazione è trascurabile e i circuiti possono essere analizzati con le approssimazioni "quasi statiche".
\subsection*{Le antenne}
Le antenne irradiano (agiscono da sorgente) e captano il campo ettromagnetico.\\\\
Campo a grande distanza di una sorgente generica
\begin{equation*}
\textbf{E}_\infty(\textbf{r})=C\frac{e^{-j\beta r}}{r}\textbf{f}(\theta,\phi)
\end{equation*}
\begin{itemize}
\item[-] i primi due fattori sono gli stessi qualunque sia la sorgente;
\item[-] il terzo dipende da dimensioni, forma e distribuzione spaziale delle correnti ed è
quindi caratteristico dell’antenna.
\end{itemize}
Le proprietà radiative di un’antenna sono descritte
\begin{itemize}
\item  ampiezza e fase dal \emph{diagramma di radiazione in campo}
\begin{equation*}
\textbf{F}(\theta,\phi)=F_\theta(\theta,\phi)\,\bm{\theta}_0+F_\phi(\theta,\phi)\,\bm{\phi}_0
\end{equation*}
\item in potenza dal \emph{diagramma di radiazione in potenza}
\begin{equation*}
P(\theta,\phi)=\frac{1}{2\eta}|\textbf{F}(\theta,\phi)|^2
\end{equation*}
il diagramma di radiazione in potenza è la densità di potenza irradiata per unità di
angolo solido ($W\cdot ster^{-1}$).
\end{itemize}
I diagrammi di radiazione dipendono dalla potenza irradiata.\\\\
Per avere un parametro che dipende solo dall'antenna, si definisce la11
\emph{funzione di direttività:}\\
diagramma di radiazione in potenza normalizzato alla densità angolare media di
potenza irradiata (trasmessa):
\begin{equation*}
D(\theta,\phi)=\frac{P(\theta,\phi)}{\frac{W_T}{4\pi}}
\end{equation*}
la \emph{direttività} di un'antenna è il valore massimo di $D(\theta,\phi)$.\\\\
Rappresentazione di $D(\theta,\phi)$ in coordinate polari
\begin{figure}[ht] 
\centering
\includegraphics[width=0.5\linewidth]{im12}
\end{figure}
\\Spesso un'antenna opera sia in trasmissione, sia in ricezione
\begin{equation*}
\text{\emph{Area equivalente}}\hspace{15mm}A_{ep}(\theta,\phi)=\frac{W_{rp}(\theta,\phi)}{\mathcal{P}}
\end{equation*}
\begin{itemize}
\item[-] $W_{rp}$ è la potenza ricevuta;
\item[-] $\mathcal{P}$ è la densità superificale di potenza incidente;
\end{itemize}
$A_{ep}$ dipende dalla polarizzazione del campo incidente.\\\\
\underline{\textbf{N.B.}} l'area eq. di un'antenna "ad apertura" è una frazione dell'area geometrica
\begin{equation*}
A_e=\eta_AA_g\hspace{15mm\eta_A\leq1}
\end{equation*}
$\eta_A$ è il rendimento di apertura, se ci fosse una distribuzione di correnti impresse uniforme sull’apertura $\,\,\Rightarrow\,\,\,\eta_A=1$.\\
Ma in pratica $0.5\leq\eta_A\leq0.8$\\\\
Introduciamo una grandezza che misura la proporzionalità tra area equivalente e direttività
\begin{equation*}
A_{e}(\theta,\phi)=\frac{\lambda^2}{4\pi}D(\theta,\phi)
\end{equation*}
\underline{\emph{Angolo a metà potenza (-3dB):}} $\,\theta_0$
\begin{equation*}
P(\frac{\theta_0}{2},\phi)=P(-\frac{\theta_0}{2},\phi)=\frac{1}{2}P(\theta_0,\phi)_{max}=\frac{1}{2}P(0,0)
\end{equation*}
\begin{equation*}
\theta_0=C\frac{\lambda}{D}
\end{equation*}
\hspace*{4mm}ci indica quant'è grande l'antenna rispetto a $\lambda$.
\begin{itemize}
\item[D:] dimensione antenna in piano contenente $\theta_0$;
\item[C:] dipende dal tipo di antenna ($C\simeq0.88\div2$ oppure $=1$ nel caso ideale di distribuzione di campo uniforme sull’apertura).
\end{itemize}
Lunghezza di lobo: $\theta_0 = \frac{\pi}{2}$.
\subsection*{Il collegamento radio}
Due antenne $A_1$ e $A_2$ a distanza "grande"
\begin{figure}[ht] 
\centering
\includegraphics[width=0.8\linewidth]{im9}
\end{figure}
\begin{equation*}
W_2=A_{e2}\mathcal{P}_1
\end{equation*}
$\mathcal{P}_1$ è la densità superficiale di potenza che $A_1$ crea sul piano di bocca di $A_2$
\begin{equation*}
\mathcal{P}_1=\frac{1}{2}E_1H_1^*=\frac{1}{2}E_{10}H_{10}^*e^{-2\int_0^Rk_j(\lambda,s)ds}=\frac{P_1}{R^2}e^{-2\int_0^Rk_j(\lambda,s)ds}
\end{equation*}
se $E_{10}$ e $H_{10}$ sono i campi in assenza di attenuazione;\\
$k_j$ è l'attenuazione specifica del mezzo attraversato
\begin{equation*}
\mathcal{P}_1=\frac{D_1W_1}{4\pi R^2}e^{-2\int_0^Rk_j(\lambda,s)ds}
\end{equation*}
coefficiente di trasmissione tra le due antenne:
\begin{equation*}
T_{12}=\frac{W_2}{W_1}=\frac{D_1A_{e2}}{4\pi R^2}e^{-2\int_0^Rk_j(\lambda,s)ds}
\end{equation*}
se le antenne sono ad apertuta, in funzione dell'area geometrica delle antenne
\begin{equation*}
T_{12}=\eta_{A_1}\eta_{A_2}\frac{A_{g1}A_{g2}}{(\lambda R)^2}e^{-2\int_0^Rk_j(\lambda,s)ds}
\end{equation*}
Il coefficiente di trasmissione
\begin{itemize}
\item cresce con le dimensioni delle antenne;
\item decresce con il quadrato della distanza;
\item cresce con la frequenza, a meno che il fattore di attenuazione, che dipede da essa, non ne alteri la dipendenza;
\item non dipende dal verso di trasmissione (se il mezzo è reciproco). 
\end{itemize}
\end{document}